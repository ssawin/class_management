\documentclass{article}


\begin{document}

{ \begin{centering}\Large
Specifications For the Class Management Tools \\ Contained in manage.pm
\end{centering}
}

The basic script takes as input a comma-separated text file.  For each
row it performs various specified actions, including processing the
data, teXing and moving files, etc and produces a list of dictionaries.  With the
resulting list of dictionaries it then outputs various sorted sublists, in
the form of individual files or emails, or a single email or file in
html, TeX, csv, or other format.  

\textbf{Input File:} The input file is a text file, separated by
returns into lines.  Line beginning with the \textbf{comment
  character} \% are treated as comments and ignored.  Apart from that
the rows are broken up entries separated by a comma surrounded by
arbitrary white space (commas inside  double quotes do not
count).  The collection of rows consist of one or more initial rows
called the \textbf{preamble}, each of which has ``='' or ``=='' as the second entry,
a single row called the \textbf{header row} which is the first row
without the ``='' in the second entry, and then some number of rows
called \textbf{record rows}.

The first entry of each preamble row and each entry of the header row
is a \textbf{header entry} consisting of a sequence of letters,
numerals and '\_','-', representing the \textbf{name} of the header
(names beginning with ``CSV-'' are used as system variables and are UAYOR), followed by an
optional special character representing the \textbf{input type} of
the header (``;'' for  LIST, ``:'' for DICTIONARY,  ``@'' for DATE,  ``!'' for CSVFILE) , followed by $0$ or
more instances of a period followed  by a name, each
representing an \textbf{action}.  The third entry in each preamble row
and all entries in each record row are \textbf{value entries}.  These
have an input type determined by the header intry in its row for
preamble rows and in its column for record entries.  No entry type is
text or numbers, the LIST input type is of the form
\{value\};\{value\};$\cdots$;\{value\}, the DICTIONARY input type is of the form
\{key\}:\{value\};$\cdots$;\{key\}:\{value\}, and the DATE input type has
various acceptable corresponding to different ways of representing
date times.
\section*{Special Characters and Strings}

In this document I will use the standard symbols for each of these
special meanings, but lets set themn at the beginning of the program
so that they can be changed at will
\begin{itemize}
  \item comment character \%
  \item assignment sting =
  \item immediate assignment string ==
  \item quote character \verb|"|
  \item list symbol ;
  \item dictionary symbol :
  \item date symbol @
  \item csv symbol !
  \item escape character \verb|\|
  \item begin evaluate \#
  \item end evaluate \#
  \item action character .
\end{itemize}
\section*{Data Types}

\begin{itemize}
\item \textbf{header} - a dictionary with keys \emph{name} pointing to
  text, \emph{input-type} pointing to text, \emph{error} pointing to a
  number, \emph{actions} pointing to a
  list with text entries, and possible other key values.  
\item \textbf{field} - a dictionary with keys \emph{header} pointing to a
  header object, \emph{value} pointing to text, \emph{errors} pointing
  to a list, \emph{immediate} pointing to a Boolean and possibly other key value pairs.  
\item \textbf{headers} - a list of header objects.
\item \textbf{record} - a list of record objects.
\item \textbf{records} - a list of records objects.  
\end{itemize}

\section*{Procedures}
 
\begin{enumerate}
\item \textbf{csv-output} -- takes a csv file as input, processes it,
  converts it to data and then outputs the data to a variety of output
  files and formats.  Specifically runs \textbf{build-data} on it with
  empty \emph{preamble-record} and \emph{preamble-datum}, runs
  \textbf{build-views} on the result, and for each view in the result
  builds an output according to the instructions in the dictionary \emph{view-globals}[\_output\_].
\item \textbf{build-views} -- takes a list of dictionaries called \emph{data}
  and a dictionary called \emph{globals} and outputs a list, each
  entry of which contains a list \emph{view-data} and a dictionary
  \emph{view-globals}.  For each dictionary in the list 
  \emph{globals}[\_views\_] it  trims \emph{data} by taking only rows
  with a given field set to True/nonzero, sorts them according to the
  instructions, and makes the result \emph{view-data}, then
  adds/replaces certain fields in \emph{globals} and calls the result \emph{view-globals}.
\item \textbf{build-data} -- takes a csv file, a record called
  \emph{preamble-record} and a dictionary called \emph{preamble-datum} and
  outputs a list of dictionaries, called  \emph{data} and a dictionary
  called \emph{globals}.
\item \textbf{is-comment} -- takes as input a line and returns $1$ if
it is a comment and $0$ otherwise.
\item \textbf{is-preamble} -- takes as input a line which is not a
  comment and returns $1$ if
it is a preamble line and $0$ otherwise.
\item \textbf{entry-to-header} -- takes as input text and returns a
  header (error set to $1$ if there is a syntax error)
\item \textbf{line-to-field}  -- takes as input one line from the
  preamble and returns a field
\item \textbf{entry-to-value} -- takes as input a field, a dictionary
  \emph{datum} and a record \emph{record}, as well as a flag for
  whether you are in the preamble.  
  \begin{itemize}
    \item if INPUT-TYPE is blank or CSV, all pieces of the entry
      string not inside quotes which are surrounded by \# get each ;
      repaced by a , and then are
      interpreted with PYTHON's EVAL command with \emph{datum} as
      globals.  If the result is a string and there are other pieces,
      it treats them as strings, concatenates and returns as a
      string.  If it is the whole entry returns the value as whatever
      type it is.  Otherwise returns a syntax error.
    \item if INPUT-TYPE is CSV, takes the string from the previous
      point, interprets as the name of a CSV file, calls
      \textbf{csv-process} with \emph{datum} and, if we are not in the
      preamble the \emph{record}.  The resulting list is the value.
    \item if the INPUT-TYPE is LIST, separates into individual strings
      by instances of ; which are not inside quotes or \# constructs.
      Each piece is then interpreted as above for a blank INPUT-TYPE,
      and then the value is a list of the results.
    \item if INPUT-TYPE is DICTIONARY, separates into individual
      strings by instances of ; which are not inside quotes or \#
      construct, interprets the contents before : as a name, and
      interprets the stuff after the : as a a blank INPUT-TYPE and
      makes the result that the value of a name value pair added to
      the dictionary which is returned as the final value.
\item if INPUT-TYPE is DATE, separates into individual strings by
  instances of + and - which are not inside quotes or \# constructs.
  If the first entry is a \# construct it is interpreted as above and
  returns a syntac error unless it is A PYTHON date-time object, if fits one of a standard set of date-time
  expression formats it is converted to a PYTHON date-time object, and
  if it is blank put NOW as a date-time object.  Each additional piece
  must look like a blank, a number, or a \# construct that evaluates
  to a number followed by one of 'cdhmsw' or their upper case versions
  or it is a syntac error.  Ignore c for now.  in each case if it is a
  number you add or subtract that many days, hours, minutes, seconds
  or weeks to the initial date time object.  If it is blank you round
  up or down acording to + or - to the end of that day, hour minute,
  second or week. Finally the value is set to a dictionary with 'dt'
  aassigned the date-time object, 'is-past' set to $1$ if it is past,
  'is-future' set to  $1$ if it is future, and if 
  'CSV-dateformats' is set to a dictionary in \emph{datum} take each
  key-value pair, interpret the value as a PYTHON date-time formatting
  string, apply it to the date-time object and assign the key and that
  reuslt as a key-value pair.
  \end{itemize}
\item \textbf{interpret-field} -- takes as input a field, a dictionary
  \emph{datum} and a record \emph{record}, as well as a flag for
  whether you are in the preamble.  First it sends this to
  \textbf{entry-to-value} to convert the entry into an object.
  Then it applies in sequence each action in the action list to this
  object (with \emph{datum} as an additional parameter).  When
  these are completed it adds \emph{header}.\emph{name} and the
  resulting object as a key-value pair to \emph{datum}. 
\item \textbf{preamble-to-record} -- takes as input the csv file, the
  \emph{preamble-datum} and the \emph{preamble-record}.  for each line
  of the preamble runs \textbf{entry-to-header}, if the assignment is
  not immediate it creates a field with that header and the third
  entry as value (if the third entry is blank, looks in
  \emph{preamble-record} for a field with the same name and uses its
  value instead.  If the assignment is immediate runs \textbf{interpret-field}  
\item \textbf{process-entry} -- takes a field, a dictionary
  (\emph{datum}), and a record (\emph{preamble-record}), interprets \emph{value} according to
  \emph{input-type}  and interpolates with
  \emph{datum}, and returns the data structure indicated by
  \emph{input-type}.  Modifies the field to indicate error information
  and the type of the result.  If \emph{value} is empty, searches
  \emph{preamble-record} for a field with the same \emph{name}, and
  substitutes its \emph{value} in and removes the field from the record before doing anything else.  
\item \textbf{process-field} -- takes a field and a hash
  (\emph{datum}) and returns the hash modified.  Runs \textbf{process-entry}
  and each of the actions (if \emph{datum.dry} is set to true then it
  does not do the actions which are not dry).
\item \textbf{process-preamble} -- runs \textbf{line-to-field} on each
  line for which \textbf{is-comment} is false while
  \textbf{is-preamble} is true.  If the field has \emph{immediate}
  set, runs \textbf{process-field} on it with \emph{preamble-datum},
  otherwise it appends the field to \emph{preamble-record}.  
\item \textbf{process-headers} -- reads the header line, runs
  \textbf{entry-to-header} on each entry, and appends it to a list of
  headers \emph{headers} which it returns.
\item \textbf{process-record} -- takes as input a line of the file, a
  hash \emph{preamble-datum} and a record \emph{preamble-record} and returns
  a hash \emph{datum}
\end{enumerate}


\section*{Syntax of the CSV File}

Below is how to check if the syntax of the CSV file is correct.  Terms
are used below for regular expression matching  that above refer
to the data structures they turn into.  
\begin{itemize}
\item \textbf{whitespace} -- space and tab
\item \textbf{name} -- *[a-z,A-Z,0-9,\_,-]
\item \textbf{full name} -- \{name\} or \{name\} followed by one of the
  following input-type characters !@;:
\item \textbf{quote} -- ``*.''
\item \textbf{field} -- \{whitespace\}\{full
  name\}*(\verb|\|.\{name\})\{whitespace\}
\item \textbf{python-code} -- the character \# surrounding valid
  python code that contains no unescaped \# chracters.
\item \textbf{text entry} -- whitespace surrounding an arbitrary
  sequence of {quote}s, {python-code} and text strings which contain
  no unescaped ,\verb|''|\# characters.  If inside a list or
  dictionary entry text must also contain no unescaped ;
\item \textbf{list entry} -- \{whitespace\}\{ python-code\}\{whitespace\}
  or \{whitespace\}\{text entry\}*(;\{text entry\})
\item \textbf{dictionary entry} -- \{whitespace\}\{python-code\}\{whitespace\}
  or \{whitespace\}\{name\}:\{text entry\}*(;\{name\}:\{text entry\})
\item \textbf{date} -- \{whitespace\} \{python-code\}\{whitespace\}
  or one of the date formats permitted
\item \textbf{date entry} -- \{date\} *((+,-)\{whitespace\}(\{number\},\{python-code\})\{whitespace\}(s,S,m,M,d,D,w,W)\{whitespace\})
\item \textbf{entry} -- (\{text entry\},\{list entry\},\{dictionary entry\},\{date entry\})
\item \textbf{comment line} -- \{linestart\}\{whitespace\}\#.*\{lineend\}
\item \textbf{preamble line} -- 
  \{linestart\}\{field\},\{whitespace\}(=,==)\{whitespace\},\{entry\}*(,\{whitespace\})\{lineend\}
\item \textbf{header line} \{linestart\}\{field\}*(,\{field\})*(,\{whitespace\})\{lineend\}
\item \textbf{record line} \{linestart\}
  \{entry\}*(,\{entry\})*(,\{whitespace\})\{lineend\}
\item \textbf{CSV text} -- *(\{ preamble line\},\{comment line\})\{header
  line\}*(\{record line\},\{comment line\})
\end{itemize}

\end{document}